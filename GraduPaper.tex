%!TEX program = xelatex

\documentclass{HustGraduPaper}
\title{移动网络数据挖掘的研究与应用}
\author{徐聪}
\date{\today}
\school{电子信息与通信学院}
\classnum{通信1403班}
\stunum{U201413500}
\instructor{王邦}

\begin{document}
    \maketitle
    \statement
    \clearpage
    \pagenumbering{Roman}

    \begin{cnabstract}{移动网络;数据挖掘;机器学习}
        这里是摘要内容。
    \end{cnabstract}
    \begin{enabstract}{Mobile network;Data mining;machine learning}
        Here is the content of abstract.
    \end{enabstract}

    \tableofcontents
    \clearpage
    \pagenumbering{arabic}

    \section{绪论}
    \subsection{选题背景与意义}
    随着我国经济社会的快速发展,移动互联网以及通信技术也进入了快速发展期。从2G时代的落后于国际发展步伐,到3G时代主导推出的TD-SCDMA标准被ITU
    确立成为3G主流制式,以及4G推进进程基本与日韩美等国家同步,到了5G时代,我国主动成立 IMT-2020 推进组,举全产业链之力,积极推进 5G 技术
    标准发展,有望实现技术与市场双引领。
    \subsection{国内外发展现状}
    \subsection{主要内容与结构}
    
    

\end{document}