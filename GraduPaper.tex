%!TEX program = xelatex

\documentclass{HustGraduPaper}
\title{移动网络数据挖掘的研究与应用}
\author{徐聪}
\date{\today}
\school{电子信息与通信学院}
\classnum{通信1403班}
\stunum{U201413500}
\instructor{王邦}

\begin{document}
    \maketitle
    \statement
    \clearpage
    \pagenumbering{Roman}

    \begin{cnabstract}{移动网络;数据挖掘;统计分析}
        这里是摘要内容。
    \end{cnabstract}
    \begin{enabstract}{Mobile network;Data mining;statistic analysis}
        Here is the content of abstract.
    \end{enabstract}

    \tableofcontents
    \clearpage
    \pagenumbering{arabic}

    \section{绪论}
    
    \subsection{选题背景与意义}
    随着我国进入经济社会高速发展的阶段,电信行业也得到了快速的发展。从上世纪九十年代的
    2G发展落后于世界;到本世纪初3G时代,由我国大唐电信主导提出的TD-SCDMA标准被ITU确立成为3G主流
    制式;进入4G时代后,我国发展时间轴基本与世界先进水平国家保持同步,由我国主导的 TDD 通信制式
    将成为 5G 主流制式。到了 5G时代,我国主动成立 IMT-2020 推进组,举全产业链之力,
    积极推进 5G 技术标准发展,有望实现技术与市场双引领。
    
    除了传统通信的2、3、4G通信网络,中国的电信运营商也在顺应移动互联网发展趋势,
    发展出了诸如IPTV、窄带物联网(NB-IoT)、集客等新兴业务。以一个三线城市市级运营商来说,
    其网络管理中心负责主要业务包含无线及有线总共十二类业务,支撑十二类业务的背后包含十三类网络、四大类设备。
    如此众多的业务伴随而来的是海量的数据:日志、参数、拓扑等等。

    面对如此海量的数据,网络维护管理人员的维护难度在不断攀升。传统的电信网络维护管理方式通常是以人工管理为主,
    运营商会向提供技术支持的公司雇佣代维人员对网络进行代理维护,而运营商内部人员则注重于对网络的管理以及核心业务上,
    这种代维模式最初是为了解决网络规模扩张、运营商间的竞争压力不断增大的问题。但随着数据网络规模的进一步扩大,网络复杂性呈
    指数型上升,但是代维人员规模不可能同步增加。除此之外,由于代维行业没有统一的行业标准,会出现代维人员业务水平参差不齐
    的情况,且在日常维护工作中仍采取较为传统的手段,导致运营商对于网络的管理维护并没有跟上网络快速发展的脚步,传统的维护
    模式已经不能适应将来的需求。

    
    阐述运营商的投入。从数据入手,实现降本增效。


    数据采集

    数据分析

    数据呈现
    
    \subsection{论文内容与成果}


    \subsection{论文结构}


    
    

\end{document} 