%!TEX program = xelatex

\documentclass{HustGraduPaper}
\title{移动网络数据挖掘的研究与应用}
\author{徐聪}
\date{\today}
\school{电子信息与通信学院}
\classnum{通信1403班}
\stunum{U201413500}
\instructor{王邦}

\begin{document}
    \maketitle
    \statement
    \clearpage
    \pagenumbering{Roman}

    \begin{cnabstract}{移动网络;数据挖掘;统计分析}
        这里是摘要内容。
    \end{cnabstract}
    \begin{enabstract}{Mobile network;Data mining;statistic analysis}
        Here is the content of abstract.
    \end{enabstract}

    \tableofcontents
    \clearpage
    \pagenumbering{arabic}

    \section{绪论}
    
    \subsection{选题背景与意义}
    随着我国进入经济社会高速发展的阶段,电信行业也得到了快速的发展。从上世纪九十年代的2G发展落后于世界;到本世纪初3G时代,由我国大唐电信主导提出的TD-SCDMA标准被ITU确立成为3G主流
    制式;进入4G时代后,我国发展时间轴基本与世界先进水平国家保持同步。4G 时代我国主导的 TDD 通信制式将成为 5G 主流制式。到了 5G
    时代,我国主动成立 IMT-2020 推进组,举全产业链之力,积极推进 5G 技术标准发展,有望实现技术与市场双引领。
    
    除了传统通信的2、3、4G通讯,  
    
    \subsection{国内外发展现状}
    \subsection{主要内容与结构}
    
    

\end{document}