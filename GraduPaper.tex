%!TEX program = xelatex

\documentclass{HustGraduPaper}
\title{移动网络数据挖掘的研究与应用}
\author{徐聪}
\date{\today}
\school{电子信息与通信学院}
\classnum{通信1403班}
\stunum{U201413500}
\instructor{王邦}

\begin{document}
    \maketitle
    \statement
    \clearpage
    \pagenumbering{Roman}

    \begin{cnabstract}{移动网络;数据挖掘;统计分析}
        这里是摘要内容。
    \end{cnabstract}
    \begin{enabstract}{Mobile network;Data mining;statistic analysis}
        Here is the content of abstract.
    \end{enabstract}

    \tableofcontents
    \clearpage
    \pagenumbering{arabic}

    \section{绪论}
    
    \subsection{选题背景与意义}

    随着我国进入经济社会高速发展的阶段,电信行业也得到了快速的发展。从上世纪九十年代的
    2G发展落后于世界;到本世纪初3G时代,由我国大唐电信主导提出的TD-SCDMA标准被ITU确立成为3G主流
    制式;进入4G时代后,我国发展时间轴基本与世界先进水平国家保持同步,由我国主导的 TDD 通信制式
    将成为 5G 主流制式。到了 5G时代,我国主动成立 IMT-2020 推进组,举全产业链之力,
    积极推进 5G 技术标准发展,有望实现技术与市场双引领。
    
    除了传统通信的2、3、4G通信网络,中国的电信运营商也在顺应移动互联网发展趋势,
    发展出了诸如IPTV、窄带物联网(NB-IoT)、集客等新兴业务。以一个三线城市市级运营商来说,
    其网络管理中心负责主要业务包含无线及有线总共十二类业务,支撑十二类业务的背后包含十三类网络、四大类设备。
    如此众多的业务伴随而来的是海量的数据:日志、参数、拓扑等等。

    面对如此海量的数据,网络维护管理人员的维护难度也在不断攀升。传统的电信网络维护管理方式通常是以人工管理为主,
    运营商会向提供技术支持的公司雇佣代维人员对网络进行代理维护,而运营商内部人员则注重于对网络的管理以及自身核心业务,
    这种代维模式最初是为了解决网络规模扩张、运营商间的竞争压力不断增大的问题。但随着数据网络规模的进一步扩大,网络复杂性呈
    指数型上升,但是代维人员规模却不可能同步增加。除此之外,由于代维行业没有统一的行业标准,会出现代维人员业务水平参差不齐
    的情况,且在日常维护工作中仍采取较为传统的手段,导致运营商对于网络的管理维护并没有跟上网络快速发展的步伐,传统的维护
    模式已经不能适应将来的需求。

    近年来,随着计算能力的提升以及信息技术的发展,大数据的概念在运营商对网络管理维护中得到了广泛传播,将大数据以及数据挖掘的技术
    引入对于电信网络的运维管理中来说是解决当前及未来需求的一种解决手段。由于运营商天生的特性,导致了它在数据量这一块拥有
    天然的优势,当前运营商的问题也是由于数据量过于巨大导致的运维管理难度上升,而大数据及数据挖掘技术的引入会将这一劣势变为
    优势,不仅能解决当前数据量过大无法对整个网络的流量、设备以及业务实现集中管控、统一编排。同时近年随着人工智能概念的火爆,
    一定程度上也是基于大数据及数据挖掘技术的成熟,因此在对电信网络数据进行大数据管理以及数据挖掘的基础上可以在合适的场景引入
    人工智能的方法,实现面向未来的智能运维。
    
    综上所述,基于移动通信网络的数据挖掘对于提升整个电信网络运维管理效率及质量具有较为实际以及深远的意义。在实际研究过程中,对运营商而言,
    大数据意味着对数据的获取、存储、分析、呈现等方式都提出了新的挑战,因此,如何对传统运维方式的这几个方面进行革新成为了运用大数据管理网络
    关键点。运营商希望将复杂的网络进行高效的管理,对于提升用户感知、网络优化、网络规划等多方面都具有实际的意义。

    \subsection{论文内容与成果}

    本文
    数据采集

    数据分析

    数据呈现

    \subsection{论文结构}
    \clearpage
    \section{相关技术概述}
    \subsection{通信网络介绍}



    
    

\end{document} 